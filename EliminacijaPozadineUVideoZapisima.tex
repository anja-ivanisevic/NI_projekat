% !TEX encoding = UTF-8 Unicode

\documentclass[a4paper]{article}

\usepackage{color}
\usepackage{url}
\usepackage[T2A]{fontenc} % enable Cyrillic fonts
\usepackage[utf8]{inputenc} % make weird characters work
\usepackage{graphicx}

\usepackage[english,serbian]{babel}
%\usepackage[english,serbianc]{babel} %ukljuciti babel sa ovim opcijama, umesto gornjim, ukoliko se koristi cirilica

\usepackage[unicode]{hyperref}
\hypersetup{colorlinks,citecolor=green,filecolor=green,linkcolor=blue,urlcolor=blue}

%\newtheorem{primer}{Пример}[section] %ćirilični primer
\newtheorem{primer}{Primer}[section]

\begin{document}

\title{Eliminacija pozadine u video zapisima\\ \small{Seminarski rad u okviru kursa\\Naučno izračunavanje\\ Matematički fakultet}}

\author{\href{mailto:mi14097@matf.bg.ac.rs}{Vesna Katanić}, \href{mailto:mi14022@matf.bg.ac.rs}{Anja Ivanišević}}
\date{14.~septembar 2019.}
\maketitle

\tableofcontents

\newpage

\section{Uvod}
\label{sec:uvod}

Algoritmi za eliminaciju pozadine u video zapisima su algoritmi koji imaju za cilj detekciju kretanja i odvajanje pozadine od objekata koji se kreću. Problem eliminacije pozadine je bitan problem obrade slika i 


\section{Primenjeni algoritmi}
\label{sec:algoritmi}

U ovom radu smo poredili tri različita algoritma za eliminaciju pozadine u video zapisima. Ti algoritmi su: \emph{Basic motion detection}, \emph{Gaussian Mixture Model} i \emph{K Nearest Neighbours}. Ideja svakog od ovih algoritama je da transformišu ulazni video u novi video u kojem će pozadina biti ofarbana u crno a objekti koji se kreću u belo, kako bi se oni izdvojili. U nastavku rada će ovi algoritmi biti predstavljeni.

\subsection{Basic Motion Detection}

\emph{Basic Motion Detection} je najjednostavniji algoritam od predstavljenih algoritama. U ovom algoritmu se polazi od pretpostavke da se video $I$ sastoji od statičke pozadine $B$ ispred koje se nalaze objekti koji se kreću. Kako bismo detektovali objekte računa se rastojanje trenutnog modela pozadine i posmatranog frejma. Na osnovu ovog rastojanja pravi se rezultujuća crno-bela slika. 

Model pozadine se ažurira na osnovu prethodnog stanja i trenutno posmatranog frejma po sledećoj formuli: 
\\
$B_{s,t+1} = (1 - \alpha) * B_{s,t} + \alpha * I_{s,t}$
\\
gde je $s$ posmatrani piksel, $t$ posmatrani frejm i $\alpha$ parametar za koju je uzeta vrednost $0.001$. Za početnu vrednost modela pozadine $B$ je uzet prvi frejm, dok je rastojanje između frejma i modela računato na osnovu Euklidskog rastojanja.



\begin{verbatim}

class SkipListNode(object):

    def __init__(self, key=None, level=None):
        self.key = key  # vrednost čvora
        self.forward =[None]*(level+1)  # niz pokazivaca
        self.d = [None]*(level+1)  # niz pomoćnih promenljivih

\end{verbatim}

Prilikom umetanja novog elementa u skip listu potrebno je da čuvamo niz $update$ u koji za svaki nivo smeštamo poslednji čvor koji smo obišli dok smo tražili mesto za naš novi element. Na slici 1. je crvenom bojom označen put i elementi koji bi u tom primeru bili u nizu $update$. Niz $update$ će nam biti potreban da bismo prevezali pokazivače i time dodali novi element u skip listu. Sada kad smo dodali novi element, upravo elementima iz niza $update$ bi trebalo ažurirati niz $d$. Takođe potrebno je i inicijalno izračunati vrednosti niza $d$ za novododati element.

Ažuriranje niza d za elemente niza update sa indeksima većim od nivoa novododatog čvora je jednostavno, potrebno je samo povećati vrednost za jedan. Međutim, u slučaju indeksa koji su manji ili jednaki nivou novog čvora, vrednost d možemo izračunati na osnovu stare vrednosti i vrednosti d za novi čvor. Potrebno je od stare vrednost d elemenata niza update oduzeti vrednost d za novi čvor i dodati jedan. Slično se radi i u slučaju brisanja elemenata iz skip liste, samo što u tom slučaju umesto sabiranja radimo oduzimanje, i obrnuto. Na ovaj način možemo jednostavnim lokalnim izmenama da ažuriramo vrednost d i time ne ugrozimo vremensku složenost algoritma.

\section{Implementacija algoritma}
\label{sec:algoritam}

Sada kada smo predstavili kako izgleda struktura skip liste, i kako se ona održava prilikom dodavanja novog elementa, ostaje nam samo da napravimo funkciju koja nalazi $k$-ti najveći element u skip listi. Pošto imamo pomoćni niz $d$ u svakom čvoru, ovo će sada biti jednostavno. 

\begin{verbatim}

    def findKLargest(self, k):
        current = self.head
        pos = 0
        for i in range(self.level, -1, -1):
            # dokle god nismo prosli dovoljno elemenata pomeramo se
            # na sledeći pokazivač i ažuriramo broj elemenata pos
            while current.forward[i] and (pos + current.d[i]) <= k:
                pos = pos + current.d[i]
                current = current.forward[i]
            if pos == k:
                # kada dođemo do k vraćamo rezultat
                return current.key

\end{verbatim}

Krećemo od prvog čvora na najvišem nivou i pomeramo se po tom nivou dokle god ne pređemo broj elemenata veći od $k$. To proveravamo sa $(pos + current.d[i]) <= k$. U svakom koraku pomeramo se na sledeći pokazivač i povećavamo broj $pos$. Na kraju ćemo doći do $k$-tog elementa i to vratiti kao rezultat. Vremenska složenost ovog algoritma za skip listu od $n$ elemenata je $O(logn)$. 


\section{Efikasnost algoritma}
\label{sec:efikasnost}

Naredne dve slike prikazuju vreme izvršavanja umetanja i brisanja novog elementa u zavisnosti od broja elemenata u skip listi izražene u sekundama. 


\begin{figure}[h!]
\begin{center}
\includegraphics[scale=0.6]{inserting.png}
\end{center}
\caption{Vreme umetanja novog elementa izraženo u sekundama u zavisnosti od broja elemenata u skip listi}
\label{skiplist}
\end{figure}



\begin{figure}[h!]
\begin{center}
\includegraphics[scale=0.6]{deleting.png}
\end{center}
\caption{Vreme brisanja elementa izraženo u sekundama u zavisnosti od broja elemenata u skip listi}
\label{skiplist}
\end{figure}


\end{document}
